% === INTRO === %

\vspace{2em}
\subsection{Introducción Teórica}

Los Métodos iterativos son procedimientos que nos permiten resolver algunos sistemas de ecuaciones lineales del tipo $Ax = b$. Contrario a los Métodos exactos como la Eliminación Gaussiana, los cuales en un número finito de pasos obtienen el resultado, estos generan una sucesión {$x^{k}$} que converge o no a la solución del sistema.

Como esquema básico, dado un $x^{0}$ inicial, se define la sucesión de nuestro Método iterativo {$x^{k}$} como:

\begin{equation}\label{sucesion}
    x^{k+1} = Tx^k + c
\end{equation},

donde $T$ es nuestra matriz de iteración. En particular, nuestro sistema va a converger en la solución del mismo para cualquier vector $x^0$ inicial si y solo si el radio espectral de la matriz de iteración T ($\rho(T)$) es menor a 1.

\begin{equation}\label{radio espectral}
    \rho(T) = max{|\lambda| : \lambda autovalor de T}
\end{equation}

En este informe, trabajaremos con los métodos de Jacobi y de Gauss-Seidel. Estos descomponen a la matriz $A$ del sistema como 

\begin{displaymath}
    A = D - L - U
\end{displaymath},

siendo $D$ la diagonal de $A$, $L$ los elementos negados por debajo de la misma y $U$ los elementos negados por encima. Luego, los esquemas para ambos métodos iterativos quedan de la siguiente forma:

$$\textit{Método de Jacobi}$$
\begin{equation}\label{jacobi}
    x^{k+1} = D^{-1} (L + U) x^k + D^{-1} b
\end{equation}

$$\textit{Método de Gauss-Seidel}$$
\begin{equation}\label{gauss-seidel}
    x^{k+1} = (D - L)^{-1} U x^k + (D - L)^{-1} b
\end{equation}

Para poder aplicar estos métodos a un sistema de ecuaciones lineales, se necesita que $A$ sea una matriz tal que no tenga elementos nulos en la diagonal. De no ser así, no se podrán calcular las inversas de $D$ y $D - L$.

% === IMPLEMENTACION === %

\vspace{2em}
\subsection{Implementación}

