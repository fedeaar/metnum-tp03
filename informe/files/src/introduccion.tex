% === INTRO === %

\vspace{1em}
\subsection{Métodos iterativos}
Los \textit{métodos iterativos} son procedimientos que nos permiten resolver algunos sistemas de ecuaciones lineales del tipo $\mathbf{A}x = b$. Contrario a los \textit{métodos exactos} ---como la \textit{Eliminación Gaussiana}--- que obtienen el resultado en un número finito de pasos, los métodos iterativos generan una sucesión $\{ x^{(k)} \}_{k \in \mathbb{N}_0}$ que, de converger, lo hace a la solución del sistema.

\vspace{1em}
Como esquema básico, dado un $x^{(0)}$ inicial, se define de manera genérica una sucesión iterativa  $\{ x^{(k)} \}_{k \in \mathbb{N}_0}$ de la siguiente manera:

\begin{equation}\label{sucesion}
    x^{(k+1)} = \mathbf{T}x^{(k)} + c
\end{equation}

\vspace{1em}
\noindent donde $\mathbf{T}$ es nuestra matriz de iteración y $c$ es un vector. En particular, $x^{(k)}$ va a converger a la solución de un sistema particular, para cualquier vector $x^{(0)}$ inicial, si y sólo si el radio espectral de la matriz de iteración \textbf{T} es menor a 1. Es decir:

\begin{equation}\label{espectral}
    \rho(\mathbf{T}) = max\{|\lambda|\ :\ \lambda \: \text{\textit{autovalor de} \textbf{T}}\}\ <\ 1
\end{equation}

\vspace{3em}
En este informe, trabajaremos con los métodos de \textit{Jacobi} y \textit{Gauss-Seidel} para la resolución de sistemas $\mathbf{A}x = b$. Estos descomponen a la matriz \textbf{A} de la siguiente forma: 

\begin{equation}
    \mathbf{A} = \mathbf{D} - \mathbf{L} - \mathbf{U}
\end{equation}

\vspace{1em}
\noindent donde $\mathbf{D}$ es la diagonal de $\mathbf{A}$, $\mathbf{L}$ contiene los elementos negados por debajo de la misma y $\mathbf{U}$ los elementos negados por encima. Luego, los esquemas para ambos métodos iterativos son los siguientes:

\vspace{1em}
\begin{center}
    \textit{Método de Jacobi}
\end{center}

\begin{equation} \label{jacobi}
    x^{(k+1)} = \textbf{D}^{-1} (\textbf{L} + \textbf{U}) x^{(k)} + \textbf{D}^{-1} b 
\end{equation}

\vspace{2em}
\begin{center}
    \textit{Método de Gauss-Seidel}
\end{center}

\begin{equation}\label{gauss-seidel}
    x^{(k+1)} = (\mathbf{D} - \mathbf{L})^{-1} \mathbf{U} x^{(k)} + (\mathbf{D} - \mathbf{L})^{-1} b
\end{equation}

\vspace{1em}
Se puede demostrar que, de converger, ambos métodos lo harán a una solución del sistema pedido. Requerimos adicionalmente, para su aplicación, que $\mathbf{A}$ sea una matriz sin elementos nulos en la diagonal. De lo contrario no se podrán calcular las inversas de $\mathbf{D}$ y $\mathbf{D} - \mathbf{L}$.





% === IMPLEMENTACION === %

\vspace{2em}
\subsection{Implementación}


% rep matrices
\subsubsection{Representación de matrices}


% jacobi
\vspace{2em}
\subsubsection{Método de Jacobi}
Definamos la siguiente función que implementa el Método de Jacobi:

\begin{align*}
    \text{\textit{jacobi}}&:\ \text{\textit{matriz}}_{n \times n}\ \mathbf{A}\ \times\ \text{\textit{vector} k}\ \times\ \text{\textit{nat} q}\ \times\ \text{\textit{real} t}\
    \longrightarrow\ \text{\textit{vector}}_x
\end{align*}


donde n es un natural. \textbf{A} es una matriz con elementos distintos a cero en la diagonal, \textit{q} es un número que indica la cantidad máxima de iteraciones a realizar y \textit{t} $\geq$ 0 representa la tolerancia mínima a partir de la que se considera la convergencia de una solución.

\lstinputlisting[mathescape=true, escapechar=@, language=pseudo, label=algo_jacobi, caption={Pseudocódigo para el Método de Jacobi.}]{files/src/.code/jacobi.pseudo}

% gauss seidel
\vspace{2em}
\subsubsection{Método de Gauss-Seidel}
Definamos la siguiente función que implementa el Método de Gauss-Seidel:

\begin{align*}
    \text{\textit{gauss\_seidel}}&:\ \text{\textit{matriz}}_{n \times n}\ \mathbf{A}\ \times\ \text{\textit{vector} k}\ \times\ \text{\textit{nat} q}\ \times\ \text{\textit{real} t}\
    \longrightarrow\ \text{\textit{vector}}_x
\end{align*}

donde n es un natural. \textbf{A} es una matriz con elementos distintos a cero en la diagonal, \textit{q} es un número que indica la cantidad máxima de iteraciones a realizar y \textit{t} $\geq$ 0 representa la tolerancia mínima a partir de la que se considera la convergencia de una solución.

\lstinputlisting[mathescape=true, escapechar=@, language=pseudo, label=algo_jacobi, caption={Pseudocódigo para el Método de Gauss-Seidel.}]{files/src/.code/gauss_seidel.pseudo}


% gauss elim
\vspace{2em}
\subsubsection{Eliminación Gaussiana}