Los \textit{métodos iterativos} representan una forma alternativa y eficiente para la resolución de sistemas lineales. Dadas las condiciones necesarias, permiten aproximar el resultado de un sistema en tiempo cuadrático, sin amortización. Es decir, logran una complejidad  menor a la que se logra con la mayoría de los otros métodos convencionales ---como lo son aquellos que requieren un paso previo de factorización---, cuyo costo no amortizado suele ser cúbico.

\vspace{1em}
En este trabajo evaluaremos la eficiencia de dos métodos iterativos: el método de \textit{Jacobi} y el método de \textit{Gauss-Seidel}, como alternativas para la resolución del algoritmo de \textit{PageRank} ---desarrollado en el tp1--- sobre una serie de casos de test. Para ello, se propondrá una posible implementación en C++ de ambos métodos y se contrastará su eficiencia con una tercera implementación basada en el método de la \textit{eliminación gaussiana} con \textit{sustitución inversa}.

A su vez, extenderemos éste análisis para evaluar el comportamiento de los tres métodos en función de la \textit{densidad} del grafo de entrada, para distintas familias de redes. 

\vspace{1em}
\noindent Palabras clave: método de Jacobi, Gauss-Seidel, Eliminación Gaussiana, PageRank.
