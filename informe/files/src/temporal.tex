\subsection{Casos de test}

Para comenzar el análisis sobre el tiempo de ejecución de cada uno de los métodos, decidimos inicialmente estudiar cómo se comportan frente a los test de la cátedra. %A partir de los resultados logramos obtener mayor claridad sobre cómo se desenvuelven. 
%Esto nos permitirá desarrollar una intuición para realizar un estudio más profundo.

\vspace{1em}
\noindent\textsc{Metodología}. Se realizó el cálculo de \textit{PageRank} sobre la implementación que utiliza la \textit{eliminación gaussiana}, con $\varepsilon = 10^{-6}$, y se calculó $||x - pagerank(g, p)||_1$ donde $x$ refiere a la solución verdadera provista por cada uno de los tests. Cabe destacar que este es un proceso determinístico. 

A partir de este resultado realizamos una búsqueda de la tolerancia ideal, para el \textit{método de Jacobi} y el \textit{método de Gauss-Seidel}, que nos permitiera calcular \textit{PageRank} con un error similar al obtenido anteriormente. Para esto, se realizó ---para ambos métodos--- una búsqueda binaria donde en cada paso alteramos la tolerancia $t$ y calculamos el resultado de \textit{PageRank}. Editamos los límites de la tolerancia acorde a si el error calculado fue mayor o menor al error que se cometió con la \textit{eliminación gaussiana}. Repetimos el proceso hasta que la diferencia entre ambas soluciones fue menor a $10^{-2} \times \varepsilon$ en norma $L_1$. %Al finalizar este procedimiento obtuvimos tolerancias tales que, al calcular el error de comparar las respuestas que conceden \textit{Jacobi} y \textit{Gauss-Seidel} con la original, sea similar, en orden de magnitud, al error entre la respuesta que retorna PageRank y la original. 

%De este modo se espera realizar una comparación de tiempo de ejecución justa, donde todos los métodos obtengan una respuesta comparable en términos del error cometido.

Dados estos parámetros, se procedió a calcular \textit{PageRank} para cada test, sobre cada implementación y medir el tiempo de ejecución de la etapa de resolución del sistema lineal asociado\footnote{Sobre esta medición importan dos aclaraciones: 1. se omitió el tiempo de construcción del sistema lineal asociado ---de orden $\Theta(n^2)$--- para poder enfocar en los algoritmos bajo estudio. 2. la medición se realizó dentro del mismo ejecutable de \textit{PageRank} para evitar el ruido que podría ocasionar medirlo de manera externa. Es decir, por medio de una llamada a un subproceso.}. Repetimos este paso diez veces para atenuar las fluctuaciones de tiempo\footnote{Para evitar disparidades sobre los resultados de este experimento, y el subsiguiente, aclaramos que estos se ejecutaron en una misma computadora.}. 

\vspace{1.5em}
\noindent \textsc{Resultados}. Procedemos a detallar los resultados en la Figura (\ref{tiempo_ej}.). 

\begin{figure}[!htbp]
    %\ContinuedFloat
    \centering
    \includegraphics[width=.9\textwidth, trim=0 30 0 30]{files/src/.media/tiempo-ejecucion.png}
    \caption{Tiempo de ejecución ---en milisegundos--- de las distintas implementaciones de \textit{PageRank} para cada uno de los tests de la cátedra (escala logarítmica).} \label{tiempo_ej}
\end{figure}


\vspace{1em}
Como se puede apreciar en el gráfico, el \textit{método de Gauss-Seidel} y el \textit{método de Jacobi} son considerablemente más rápidos cuando el tamaño de la matriz es grande. Para el test \textit{30\_segundos}, estos métodos tardaron alrededor de dos milisegundos, mientras que la \textit{eliminación gaussiana} tardó más de diez segundos. Sin embargo, cuando el tamaño de la matriz es chico, notamos que la \textit{eliminación gaussiana} es más rápida. 

\vspace{1em}
Por su parte, los métodos iterativos tienen una velocidad muy similar. El \textit{método de Gauss-Seidel} fue apenas más rápido que el \textit{método de Jacobi} en todos los casos evaluados.





% DENSIDADES
\vspace{2em}
\subsection{En función de la densidad del grafo de entrada}

Más allá de los tests de la cátedra, es importante evaluar el comportamiento de los métodos propuestos en función de sistemas alternativos.
%Podriamos esperar que el rendimiento de nuesta solución varie con el comportamiento inherente a cada grafo. 

\vspace{1em}
En particular, es esperable que el rendimiento de \textit{PageRank} varíe según las cualidades inherentes al tipo de grafo sobre el que se lo ejecute.
Para poner esto en práctica vamos a generar un conjunto de entradas y analizar su comportamiento en virtud de su densidad.
Es decir, dadas ciertas familias de grafos (ver figura \ref{familias}.), produciremos instancias aleatorias de ellas con diferentes proporciones de aristas ---links--- entre sus nodos.
% Buscamos obtener una mejor idea de como se comportan nuestros algoritmos.

\begin{figure}[!htbp]
    \centering
    \subfloat[Aleatorias]{\includegraphics[width=0.3\textwidth]{files/src/.media/familia_aleatorio.png}}
    \subfloat[Sumidero]{\includegraphics[width=0.3\textwidth]{files/src/.media/familia_red_sumidero.png}}
    \subfloat[Completas]{\includegraphics[width=0.3\textwidth]{files/src/.media/familia_todo_con_todo.png}}
    \newline
    \subfloat[Estrella]{\includegraphics[width=0.3\textwidth]{files/src/.media/familia_uno_con_todo.png}}
    \subfloat[Viborita]{\includegraphics[width=0.3\textwidth]{files/src/.media/familia_viborita.png}}
    \vspace{1em}
    \caption{Ejemplos de las familias de grafos generadas.} %para el análisis de densidad.}
    \label{familias}
\end{figure}

% \noindent\textsc{Metodología}.
% En seguimiento de la sección anterior, atentamos a reconstruir la misma metodología.
% Para cada familia de grafos generamos 7 instancias aleatorias, con densidad creciente $d \in \{0.01, 0.05, 0.1, 0.3, 0.5, 0.9, 1\}$.
% Estas muestras son aleatorias en vista de que los $N$ nodos que participan de dicha construcción se seleccionan al azar.
% Dado un $N$ grande podemos prever una descripción ejemplar de cada categoría. %exceptuando casos extraordinarios donde el arreglo de valores en la matriz de entrada sea especial.
% Para \textit{PageRank} usamos nuevamente la tolerancia $\varepsilon = 10^{-6}$.
% En búsqueda de una tolerancia justa para los otros dos métodos, desarrollamos el binary-search haciendo uso del error relativo $||Ax - x||_1$,
% para el resultado $x$ de cada recursión. 

% Con $N = 3000$, se itera por cada $d$ y por cada familia de grafos. En cada caso, experimentamos con cada uno de los tres métodos, 10 veces cada uno.
% De estas 1050 muestas construimos los resultados. 

\noindent\textsc{Metodología}. Para cada familia de grafos generamos siete instancias aleatorias, con densidad creciente $d\ \in\ \{0.01,\ 0.05,\ 0.1,\ 0.3,\ 0.5,\ 0.9,\ 1\}$.
Estas muestras son aleatorias en tanto los $n$ nodos que participan se seleccionan al azar por medio del algoritmo $sample$ de la biblioteca estándar $random$ de $python$.
Dado un $n$ grande podemos prever una descripción ejemplar de cada categoría bajo estudio. 

En concordancia con la sección anterior, evaluamos la implementación de \textit{Eliminación Gaussiana} de \textit{PageRank} con $\varepsilon = 10^{-6}$ y repetimos el proceso de búsqueda binaria, en base al error relativo $||\mathbf{A}x - x||_1$\footnote{Notamos que esta medida es válida ya que \textit{PageRank} equivale al cálculo del autovector asociado al autovalor en módulo máximo, $|\lambda| = 1$, de \textbf{A}. Para más detalle ver el $tp1$ y $tp2$.}, ---donde \textbf{A} es la matriz asociada al sistema lineal de \textit{PageRank} y $x$ es la solución obtenida---. %para obtener un valor de tolerancia a utilizar en las implementaciones de \textit{Gauss-Seidel} y \textit{Jacobi} que permita obtener un resultado comparable en error al que provee la \textit{Eliminación Gaussiana}.

Se utilizó $n = 1000$ y se iteró por cada $d$ y por cada familia de grafos. En cada caso, y para cada implementación, se obtuvieron los parámetros correctos y se midió el tiempo de ejecución de la etapa de resolución del sistema lineal asociado. Se repitió el experimento diez veces para capturar su variabilidad. Además, se controló el \textit{valor p} ($p = 0.5$).

\vspace{1.5em}
\noindent\textsc{Resultados}. A simple vista, confirmamos la conclusión de la sección previa: los métodos iterativos son considerablemente más rápidos que la \textit{eliminación gaussiana}.

\vspace{1em}
El caso de las redes aleatorias (figura \ref{densidad_aleatorio}.) pone en manifiesto el comportamiento generalmente esperado: 
el tiempo de ejecución crece en proporción a la densidad del grafo. Los tres métodos mantienen prolijamente su eficiencia relativa.
Este comportamiento se refleja también en el resto de las figuras, si bien pierden esta nitidez.
Es decir, la precedencia ---en tiempo de ejecución--- de cada método se mantiene en general, pero no siempre es tan clara como en el caso de los grafos aleatorios.

\begin{figure}[!htbp]
    \centering
    \includegraphics[width=.9\textwidth, trim=0 0 0 30]{files/src/.media/densidad_aleatorio.png}
    \caption{\textbf{Aleatorio}. Tiempo de ejecución para el conjunto \textit{aleatorio} de grafos, con $n = 1000$ y $p = 0.5$, en función de la densidad de conexiones, para cada implementación de \textit{PageRank}.}
    \label{densidad_aleatorio}
\end{figure}

\vspace{1em}
Las redes \textit{sumidero} (figura \ref{densidad_red_sumidero}.) y los grafos \textit{estrella} (figura \ref{densidad_uno_a_todos}.), en particular,
muestran tiempos de ejecución muy similares para los métodos de \textit{Jacobi} y \textit{Gauss-Seidel}.
%En ambos, el crecimiento de la densidad afecta en gran parte a la \textit{eliminación gaussiana}, pero sólo levemente a los métodos iterativos.

También, cabe destacar la variación inesperada para matrices particularmente ralas, como podemos notar en las redes \textit{sumidero} y \textit{viborita} (figura \ref{densidad_viborita}.).
Por el otro lado, para grafos \textit{completos} (figura \ref{densidad_todo_con_todo}.), la distinción es clara: \textit{Gauss-Seidel} es el más rápido, siempre.

\begin{figure}[!htbp]
    \centering
    \includegraphics[width=.9\textwidth, trim=0 30 0 30]{files/src/.media/densidad_red_sumidero.png}
    \caption{\textbf{Sumidero}. Tiempo de ejecución para el conjunto de redes \textit{sumidero}, con $n = 1000$ y $p = 0.5$, en función de la densidad de conexiones, para cada implementación de \textit{PageRank}.}
    \label{densidad_red_sumidero}
\end{figure}

\begin{figure}[!htbp]
    \centering
    \includegraphics[width=.9\textwidth, trim=0 30 0 30]{files/src/.media/densidad_todo_con_todo.png}
    \caption{\textbf{Completo}. Tiempo de ejecución para el conjunto de grafos que conecta cada nodo con todos los demás, con $n = 1000$ y $p = 0.5$, en función de la densidad de conexiones, para cada implementación de \textit{PageRank}.}
    \label{densidad_todo_con_todo}
\end{figure}

\begin{figure}[!htbp]
    \centering
    \includegraphics[width=.9\textwidth]{files/src/.media/densidad_uno_a_todos.png}
    \caption{\textbf{Estrella}. Tiempo de ejecución para el conjunto de grafos que conecta un nodo con todos los demás, con $n = 1000$ y $p = 0.5$, en función de la densidad de conexiones, para cada implementación de \textit{PageRank}.}
    \label{densidad_uno_a_todos}
\end{figure}

\begin{figure}[!htbp]
    \centering
    \includegraphics[width=.9\textwidth]{files/src/.media/densidad_viborita.png}
    \caption{\textbf{Viborita}. Tiempo de ejecución para el conjunto de redes \textit{viborita}, con $n = 1000$ y $p = 0.5$, en función de la densidad de conexiones, para cada implementación de \textit{PageRank}.}
    \label{densidad_viborita}
\end{figure}

\vspace{1em}
Lo que se puede concluir de estos resultados es la pertinencia de la estructura del grafo que se está analizando.
El comportamiento temporal de los métodos propuestos es muy dependiente de la misma, de tal manera que
producir una estimación de la eficiencia de \textit{PageRank} podría requerir un estudio puntual al grafo de entrada.
No obstante, destacamos, como regla de oro, el predominio del \textit{método de Gauss-Seidel} sobre los otros dos.
Donde, incluso si no es el más rápido, exhibe poca desventaja sobre el resto.

